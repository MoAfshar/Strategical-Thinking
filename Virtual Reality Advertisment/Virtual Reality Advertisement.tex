% !TEX TS-program = xelatex
% !TEX encoding = UTF-8

% This is a simple template for a XeLaTeX document using the "article" class,
% with the fontspec package to easily select fonts.

\documentclass[11pt]{article} % use larger type; default would be 10pt

\usepackage{fontspec} % Font selection for XeLaTeX; see fontspec.pdf for documentation
\defaultfontfeatures{Mapping=tex-text} % to support TeX conventions like ``---''
\usepackage{xunicode} % Unicode support for LaTeX character names (accents, European chars, etc)
\usepackage{xltxtra} % Extra customizations for XeLaTeX
\usepackage{titling}
\usepackage[parfill]{parskip}
\setlength{\droptitle}{-10em}
\usepackage{emptypage}
\usepackage{gensymb}

\usepackage[font=scriptsize,labelfont=bf]{caption}
%\setsansfont{Deja Vu Sans}
%\setmonofont{Deja Vu Mono}

% other LaTeX packages.....
\usepackage{geometry} % See geometry.pdf to learn the layout options. There are lots.
\geometry{a4paper} % or letterpaper (US) or a5paper or....
%\usepackage[parfill]{parskip} % Activate to begin paragraphs with an empty line rather than an indent

\usepackage{graphicx} % support the \includegraphics command and options

\title{Virtual Reality: The Future of Advertising}
\author{Mo Afshar - Intended for Wired Magazine}
%\date{} % Activate to display a given date or no date (if empty),
         % otherwise the current date is printed 

\begin{document}
\maketitle
Nowadays Virtual Reality (VR) systems have drawn significant attention by researchers, companies and in the last year to ordinary consumers. More so advertising has also becoming more digital than ever and because of the huge digital shift during the last half of a century, different techniques and measures are being experimented for the future of advertisement. We are constantly exposed to advertisement where we consciously and sometimes unconsciously make significant decisions. Whether if it is what we are going to eat the next day to rather more important adoptions such as choosing the path of our career. In this article, we will begin to explore the future of advertisement through the realm of virtual reality and going into more depth about how this technology can revolutionise the advertisement industry. Additionally, we will explore the differences between the traditional methods in contrast of this new technology and how they can affect us in our everyday lives.  
\par

\section{What is Virtual Reality?}
According to the Journal of Modern Engineering Research VR “is a term that applies to computer-simulated environments that can simulate physical presence in places in the real world, as well as in imaginary worlds”. VR is a believable, interactive world which can enable you to feel completely immersed within the environment you are in, both mentally and physically. 
\par

VR can essentially be put into four categories. Firstly, It should be believable; in other words, it has to feel like you are in a virtual world or the illusion of VR will disappear. It should be interactive and immersive; interactivity adds significance more to immersion and the experience the user is having. Finally, a VR world must be explorable; this is essential as it allows the user to explore and interact with the detail and different elements of the environment which further increases engagement and enthusiasm for the user.  
\par

\section{Technology of Advertisements Now}
Advertisement has evolved into an extremely vast form of communication for businesses to get a message to a consumer. The oldest and the most traditional way of advertising which still is into play, is print advertising. Also, called the periodical advertising, if it is in a magazine or a newspaper, which come out in regular intervals, then it’s a periodical advertising. For decades, this method of advertising were the gold standards for advertisers. Additionally, brochures, leaflets, flyers and so on are more intimate and long-form way of engaging the consumer. Outdoor advertising that describes any type of advertising that is outside of the home, including billboards and bus shelter posters, are a similar type of advertisement but they appeal to a more general audience or to a location which has the highest viewership reach. 
\par

However, print advertisement is taking a back seat to the many digital forms of advertisement now available to the marketers. Advertisement through the internet (World Wide Web) are almost everywhere. Adverts are presented to many websites, social media sites and more, as they are the primary revenue driver for the internet. Additionally, by forming partnerships between different service holders, the fastest and easiest way to reach millions of potential customers is online. Moreover, mobile phones and tablets or any other portable electronic device that can connect to the internet are another emerging advertising platform.
\par

\section{VR Technology}
VR requires much more resources than a standard desktop systems or your traditional television. It requires an input and output hardware devices as well as special software is also required. Input devices such as the head mount display (HMD), trackers and a 3D mouse are ways a user communicates with the computer. These devices coordinate properly together to make the user’s environmental control as natural as possible. The output devices are responsible for the presentation of the virtual environment and its phenomena to the user. These can include, haptic, visual and auditory displays to enhance the information. Finally, software undertakes an important role beyond input and output. It is the software’s responsibility for the management of such devices alongside the analysis of incoming data and generating the appropriate feedback.
\par

Visual information is the most important aspect in creating an illusion of immersion within a virtual world. Preferably, we would like to generate feedback as close as possible to the human visual system or exceeding it. Below we will view some of the important visual technologies within VR that enables it to become as immersive as possible. 
\par

Field of view (FOV) is an important factor within VR. The human eye has both vertical and horizontal FOV, the vertical range equals to around 150\degree  as it is limited by the cheeks and eyebrows. The horizontal view is limited, and equal to 150\degree:60\degree  towards the nose and 90\degree  to the side which gives a 180\degree  of total horizontal viewing range. A typical HMD can support 40\degree to 60\degree  field of view and some other screens which use wide open optics can support up to 140\degree  which is a significant difference to the 48\degree  FOV of a 21” monitor screen.
\par

Today’s current technology supports fully for the removal of temporal resolution of the eye which is the flickering phenomena perceived by humans when viewing a screen that is updated by repeated impulses. By having low refresh rates the perception of flickering occurs especially for big displays, to avoid this problem the critical fusion frequencies must be used which is 15Hz for small screens and 50Hz for larger screens. Current market monitors and screens support 76Hz refresh rates and more and for modern LCD screens this problem does not occur due to the constant screen being updated. 
\par

Many applications of VR require an accurate perception of depth, which relies together on the perception of distances and sizes of objects. The bits of information that the brain extracts from the images the eyes see from the actual state of the eyes is called depth cues. Depth cues are generally categorised in two groups: physiological which includes stereopsis, accommodation and convergence as well as psychological which includes overlap, object size, motion parallax or height in visual field. Most of these cues must be generated without providing contradictory cues to the user for a useful depth perception. 
\par

\section{VR and Marketing Today}
So far we have had a look at VR in general and some of the important specific technologies required to create the best experience for the user which currently is unavailable or less effective in other forms such as the television. In this section, we will briefly discuss the current developments of VR in markets today. Entertainment marketers are a natural fit for VR as seen by the company HBO. A 90-second experience of the “700-foot ice wall” from the popular show “Game of Thrones” which included sound effects to reproduce wind and shaking of the floor as they arise in a virtual elevator was seen by 83,000 people in cities around the world. 
\par

It has been stated that a 90-second computer generated imagery (CGI) TV spot roughly costs around \$1 million to \$2 million as stated by the Framestore’s head of digital Mike Woods. According to adage.com’s interview with Mr. Woods, the cost of VR works is relatively much lower despite the newness of this work. It has been stated the company has been enquired with many VR type projects since the release of the “Game of Thrones” VR experience, some of which are with marketers. 
\par

\section{Seeing is Believing}
For now, most projects are constrained to venues where companies can make VR headsets and custom experiences for their audiences. However, in the near future companies such as Sony, Samsung and Oculus will be releasing HMD’s for home use at affordable prices and they can be used with the 30 or 40 million home-gaming systems. As this technology develops to become more immersive by the technological advancements in vision as well as VR becoming more accessible to the public, VR advertisement will have its breakthrough. 
\par

Ultimately, the best applications of VR in marketing are still to be uncovered. However, VR creates a platform which is completely different to any other form of media that is truly immersive and gives the user a sense of presence and existence; this is an entirely a new way of storytelling and communication. 

\end{document}
